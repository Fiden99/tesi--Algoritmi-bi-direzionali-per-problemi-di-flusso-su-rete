\documentclass{article}
\usepackage[utf8]{inputenc}
\usepackage{amsmath}

\title{Presentazione algoritmo di ricerca bidirezionale di flusso senza ottimizzazione con stesso valore di nodi esplorati da source e da sink}
\author{Filippo Magi }

\begin{document}
\maketitle

\section{strutture dati}

\subsection{BiEdge}

L'arco BiEdge che collega due nodi, che ha le informazioni sulla capacità residua e sulla quantità di flusso inviata, oltre a un booleano per capire se durante l'invio del flusso deve inviarlo o ritirarlo.

\subsection{Node}

Il nodo contiene le informazioni sugli archi a lui collegati con una lista di BiEdge.
Contiene un intero Label, che rappresenta la distanza tra quel nodo e $s$ o $t$, a seconda da quale parte è stato esplorato.
Tre booleani, il primo mi rappresentano se è stato esplorato da $s$ o da $t$, il secondo se l'arco che lo collega con il nodo precedente (e quindi più vicino a $s$/$t$) ha capacità =  0 e l'ultimo mi indica se è stato esplorato o meno (nell'iterazione corrente, se fosse necessario resettarlo)
Inoltre ci sono le informazioni di indirizzamento (nextNode e nextEdge per i nodi esplorati da $t$, previousNode e previousEdge per i nodi esplorati da $s$).

\subsection{Graph}
Graph è rappresentato da due insiemi, uno contenente i nodi esplorati da source, l'altro contenente i nodi esplorati da sink.

\section{Descrizione}

Effettuo la bfs, mi restituisce un nodo che, tramite le informazioni
di indirizzamento, riesco a ottenere un percorso.
Calcolo il flusso inviabile attraverso quel percorso.
Invio il flusso nel percorso, salvando dove si creano i colli di bottiglia.
Nel caso non mi venga restituito un nodo (null) o il flusso inviabile sia zero, termino l'algoritmo.
\subsection{Bfs}
Ricevo in input due booleani che mi indicano se devo esplorare solo la parte esplorata da source, da sink o se devo esplorarle entrambe.
Cancello le informazioni in quella parte di grafo e accodo nell'apposita coda il nodo iniziale (cioè il nodo sorgente $s$ o il nodo $t$).
Finché entrambe le code non sono vuote, proseguo come segue
\begin{enumerate}
    \item se la coda source non è vuota, mentre o la coda di archi di source è vuota oppure sia la coda di sink sia la coda di archi di sink sono vuote, proseguo, altrimenti passo al punto 4
    \item seleziono il valore $elementSource$ attraverso la dequeue della coda di source
    \item per ogni arco di $elementSource$, seleziono quelli con nodi esplorabili e li inserisco nella coda degli archi di source
    \item ripeto i precedenti tre punti per la coda di sink, salvandomi $elementSink$
    \item finché entrambe le code di archi non sono vuote, esploro i nodi selezionando un arco alla volta, altrimenti torno al punto 1
    \item nel caso in cui trovasi un nodo esplorato dall'altra parte, significa che ho trovato un percorso, aggiorno i dati e restituisco il valore.
\end{enumerate}
Nel caso non trovi un percorso, restituisco null
\end{document}