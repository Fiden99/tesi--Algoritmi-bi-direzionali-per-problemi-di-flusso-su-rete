\documentclass{article}
\usepackage[utf8]{inputenc}
\usepackage{algorithm}
%\usepackage{algorithmic}
\usepackage{algcompatible}
\usepackage{amsmath}

\title{Presentazione algoritmo di ricerca di flusso massimo con propagazione della malattia, esplorata tramite "propagazione dei nodi"}
\author{Filippo Magi }

\begin{document}

\maketitle
\section{Strutture dati}

\subsection{BiEdge}
BiEdge rappresenta l'arco che collega tra di loro due nodi.
Ha i seguenti attributi
\begin{itemize}
    \item \textbf{Flow}, rappresentato nello pseudo-codice come funzione $f(e)$, dove $e$ è un arco generico, rappresenta la quantità di flusso inviata in quell'arco
    \item \textbf{Capacity},  rappresentato nello pseudo-codice come funzione $u_f(e)$, dove $e$ è un arco generico, rappresenta la capacità residua di quell'arco
    \item \textbf{Reversed}, utile durante il l'invio del flusso, mi indica se devo inviare del flusso oppure diminuirlo %//TODO ricordarsi di capire quale 
\end{itemize}
BiEdge ha un metodo a lui associato : \textbf{SendFlow}.

SendFlow riceve in input il valore del flusso che deve inviare e va a aumentare e diminuire il valore del flusso e della capacità residua, nel caso in cui reversed sia falso, altrimenti va a aumentare la capacità residua e diminuire il flusso inviato.

Ritorna due valori booleani, il primo mi indica se la capacità residua è pari a 0, quindi che quell'arco è diventato un arco bottleneck.

Il secondo mi indica che il valore del flusso o della capacità sono negativi, quindi c'è stato un errore, in quel caso aggiorno il valore del flusso da inviare per tutti i nodi e correggo l'errore dove ho già inviato il flusso.

\subsection{Node}
Node rappresenta, appunto, i nodi.
Un generico nodo $nodo$ è rappresentato dai seguenti attributi
\begin{itemize}
    \item \textbf{Name}, usato per comprendere di quale nodo si sta parlando %da valutare se va rimosso in fase finale
    \item \textbf{Edges}, è la lista di archi che entrano ed escono $nodo$, nello pseudo-codice è indicato anche come funzione $\delta(node)$, inoltre vengono usati anche $\delta^+(node)$ per rappresentare i nodi uscenti da $node$ e $\delta^-(node)$ per rappresentare i nodi entranti in $node$
    \item \textbf{SourceSide}, è un valore booleano che mi indica se $nodo$ è stato esplorato partendo dal nodo sorgente $s$ o dal nodo destinazione $t$
    \item \textbf{Label}, rappresenta la distanza tra $nodo$ e il nodo sorgente $s$(nel caso in cui il nodo sia SourceSide) o dal nodo destinazione $t$
    \item \textbf{PreviousNode}, rappresenta il nodo predecessore, cioè, il nodo,esplorato a partire da $s$, che è stato usato per esplorare $nodo$, questo attributo è usato solo nei nodi esplorati a partire da $s$ e nei nodi di confine, cioè i nodi dove i nodi esplorati da $s$ e quelli esplorati da $t$ si incontrano
    \item \textbf{PreviousEdge}, arco che collega $nodo$ e il suo PreviousNode, valgono le stesse valutazioni di quest'ultimo
    \item \textbf{NextNode}, rappresenta il nodo successore, cioè il nodo,esplorato a partire da $t$, che è stato usato per esplorare $nodo$, questo attributo è usato solo nei nodi esplorati a partire da $t$
    \item \textbf{NextEdge}, rappresenta l'arco che collega $node$ con il suo NextNode, valgono le stesse valutazioni di quest'ultimo
    \item \textbf{FlussoPassante} o \textbf{InFlow}, rappresenta la quantità di flusso che, con le informazioni ricevute fino a quel momento, è possibile inviare  attraverso il percorso descritto attraverso NextNode e PreviousEdge
    \item \textbf{SourceValid} rappresenta se $nodo$ è considerato valido nella parte esplorata a partire da $s$, quindi anche i nodi di confine
    \item \textbf{SinkValid} rappresenta se $nodo$ è considerato valido nella parte esplorata a partire da $t$
\end{itemize}
Con nodo valido si intende che nel percorso descritto non vi sono i cosiddetti archi bottleneck, cioè archi dove non è più possibile inviare il flusso.
I metodi a lui associati sono metodi per cambiare i valori e per inserire gli archi, nello pseudo-codice sono riassunti come segue
\begin{itemize}
    \item \textbf{update}, dati un nodo e un arco, dalle informazioni ivi contenute, aggiorno il valore di SourceSide, Label (attraverso il metodo di Graph ChangeLabel) e FlussoPassante. Nel caso in cui il nodo dato sia stato esplorato a partire da $t$ aggiorno NextEdge, NextNode e SinkValid, in caso contrario aggiorno PreviousEdge, PreviousNode e SourceValid
    \item \textbf{updatePath}, dati un nodo e un arco,  aggiorno i valori di PreviousEdge, PreviousNode e SourceValid, inoltre lo dichiaro nodo di confine (utilizzando il metodo di Graph AddLast)
    \item \textbf{AddEdge}, aggiunge l'arco ricevuto in input in Edges
    \item \textbf{Reset}, azzero il  valore di FlussoPassante di $node$, in maniera tale che possa essere nuovamente esplorato
\end{itemize}


\subsection{Graph}
Graph rappresenta il grafo, quindi l'insieme dei nodi e, di conseguenza, degli archi.
È rappresentato come segue
\begin{itemize}
    \item \textbf{LabeledNodeSourceSide}, è una lista di insiemi, dove ogni insieme contiene i nodi con una certa label, vi sono contenuti tutti i nodi esplorati a partire da $s$. Durante la fase di creazione del grafo, tutti i nodi tranne $t$ sono qui inseriti, in attesa che vengano esplorati per la prima volta.
    \item \textbf{LabeledNodeSinkSide} , è una lista di insiemi, dove ogni insieme contiene i nodi esplorati a partire da $t$ con una determinata label. Il nodo $t$ è qui inserito durante la fase di creazione del grafo.
    \item \textbf{LastNodesSinkSide}, è un insieme di nodi che contiene i nodi che abbiamo chiamato "di confine".
    \item \textbf{LastNodesSourceSide}, è un insieme di nodi che contengono i nodi esplorati a partire da $s$ che sono collegati con i cosiddetti nodi di confine
\end{itemize}
I metodi contenuti in Graph sono i seguenti:
\begin{itemize}
    \item \textbf{ResetSourceSide}, prende in input una intero, per tutti i nodi con label pari o superiore che sono stati esplorati a partire da $s$ (tranne i nodi di confine), indico che hanno FlussoPassante pari a 0, intendendo che non sono stati esplorati
    \item \textbf{ResetSinkSide},prende in input una intero, per tutti i nodi con label pari o superiore che sono stati esplorati a partire da $t$ , indico che hanno FlussoPassante pari a 0, intendendo che non sono stati esplorati.
    \item \textbf{ChangeLabel}, prende in input il nodo da spostare di label, un booleano che mi indica se è stato esplorato a partire da $s$ o da $t$, e la label. Rimuovo dall'opportuno insieme il nodo indicato e lo inserisco in quello indicatomi, aggiornando i dati del nodo stesso
    \item \textbf{AddLast}, prende in input un nodo che deve essere stato esplorato a partire da $t$, lo inserisce in LastNodesSinkSide e inserisce tutti i nodi a lui collegati esplorati a partire da $s$ in LastNodesSinkSide.
\end{itemize}
Nello pseudo-codice si è preferito indicare con formule matematiche le azioni da avvenire, ma si rifanno a queste.

\section{Descrizione algoritmo}

\subsection{FlowFordFulkerson}
Ricevuta in input una rete, inizializzo inserendo nelle due pile di nodi $vuotiSource$ e $vuotiSink$ i nodi $s$ e $t$.

Ripeto i seguenti passi finché o non trovo un percorso o il flusso inviabile attraverso quel percorso è pari a 0:
\begin{enumerate}
    \item inizio una ricerca di un nuovo cammino attraverso \textbf{DoBfs}
    \item salvo il nodo di confine $n$ e il flusso inviabile $s$ da \textbf{DoBfs}
    \item cancello le informazioni contenute in $vuotiSource$ e in $vuotiSink$
    \item retrocedo da $n$ verso $s$ inviando il flusso $f$ (secondo le indicazioni dell'arco), attraverso il percorso descritto da PreviousNode
    \item nel caso un arco,dopo l'invio del flusso, abbia capacità pari a 0, lo inserisco nella pila $vuotiSource$
    \item da $n$ avanzo fino a $t$, inviando il flusso $f$ (secondo le indicazioni dell'arco), attraverso il percorso descritto da NextNode
    \item nel caso un arco,dopo l'invio del flusso, abbia capacità pari a 0, lo inserisco nella pila $vuotiSink$
    \item aggiorno la quantità di flusso inviata
\end{enumerate}

\subsection{DoBfs}
Ricevo in input il grafo, la pila di nodi senza capacità ottenuti durante l'ultimo invio del flusso sia della parte di source $noCapsSource$ sia della parte di sink $noCapsSink$.

\begin{enumerate}
    \item Inizializzo tre code : $codaSource$, $codaSink$, $malati$, e due booleani : $sourceRepaired$, $sinkRepaired$, che, per il momento, mi indicano se la pila corrispondente è vuota o meno;
    \item cerco di riparare ogni nodo presente in $noCapsSource$, aggiornando il FlussoPassante tra il nodo precedentemente esplorato e quello che quello che dovrò provare a riparare. Inserisco tutti i nodi non riparati in $malati$,e salvo mi salvo il primo nodo non riparato in $noCapSource$;
    \item nel caso sia riuscito a riparare tutti i nodi e $noCapsSink$ è vuota, tra i nodi di confine cerco un percorso con l'ultimo nodo riparato, nel caso lo trovi con flussoPassante positivo, significa che ho trovato un cammino e restituisco i dati;
    \item per ogni nodo contenuto in $malati$, eseguo sickPropagation (della parte opportuna);
    \item se sickPropagation mi restituisce un nodo di confine, e tale nodo può anche raggiungere $t$, indico che ho trovato un nuovo cammino e restituisco i valori;
    \item se non ho nodi in $noCapsSink$(quindi $sinkRepaired = true$),per ogni nodo di confine che è SourceValid e SinkValid (con ulteriori controlli nel codice), cerco di raggiungere il nodo source, se lo raggiunge, ho trovato un nuovo cammino;
    \item aggiorno sourceRepaired con il valore e inizializzo la coda $codaSource$, aggiungendo o il nodo sorgente $s$, o i nodi appartenenti a $LastNodesSourceSide$, oppure i nodi con label inferiore di uno a $noCapSource$, in tal caso per ogni nodo con label pari o superiore, esplorato a partire da $s$, lo indico come non esplorato, attraverso la funzione Reset;
    \item eseguo gli ultimi sei punti per i nodi presenti in $noCapsSink$, attraverso le variabile dedicate ai nodi esplorati a partire da $t$; % va bene così o è preferibile scriverli in maniera estesa
    \item inizio la ricerca bidirezionale, che finirà o quando ho trovato un cammino, o quando entrambe le code $codaSource$ e $codaSink$ saranno vuote;
    \item se è presente almeno un elemento in $codaSource$ e la parte di source non è già stata totalmente riparata ($sourceRepaired = true)$, estraggo un nodo $element$ da quest'ultima coda;
    \item nel caso $element$ sia un nodo valido e già esplorato, esploro tutti i suoi nodi adiacenti;
    \item nel caso il nodo adiacente che sto esplorando non sia stato esplorato precedentemente e non è stato precedentemente esplorato da $t$, aggiorno i suoi dati (funzione update) e lo inserisco in $codaSource$;
    \item nel caso il nodo adiacente che sto esplorando sia già stato esplorato a partire da $t$ e sia valido, indico che ho trovato un cammino,quindi aggiorno i dati (updatePath) e lo restituisco;
    \item nel caso il nodo adiacente che sto esplorando sia stato esplorato a partire da $t$,non potrà essere esplorato durante questa "iterazione"(dato che $ sinkRepaired = true$) e ha flussoPassante = 0, allora cerco di ripararlo,senza la limitazione della label, nel caso riesca ho trovato un nuovo cammino, altrimenti indico che la parte di sink non è stata riparata e inizializzo la coda a nodo $t$;
    \item procedo a esplorare la parte di Sink, controllando che non sia stata già totalmente riparata e che ci sia almeno un nodo in $codaSink$;
    \item estraggo dalla coda il nodo $element$, controllo che sia stato esplorato ($flussoPassante> 0)$, sia SinkValid e che sia stato esplorato a partire da $t$ ($\neg SourceSide)$, in caso positivo procedo ad analizzare i suoi nodi adiacenti;
    \item nel caso il nodo adiacente che sto esplorando sia già stato esplorato ($flussoPassante> 0$) ed è sia stato esplorato da $s$ sia SourceValid, indico che ho trovato un percorso,quindi aggiorno i dati (updatePath) e restituisco il nodo $element$;
    \item nel caso il nodo adiacente che sto esplorando non sia già stato esplorato, aggiorno i dati (metodo update).
\end{enumerate}

\subsection{RepairNode}
Questa funzione mi permette di cercare di riparare un nodo $n$, con una indicazione che mi indica, nel caso sia un nodo di confine, se devo considerare i nodi esplorati da $s$ o quelli esplorati da $t$ .
Nel caso riceva in input il nodo $s$ o il nodo $t$, indico fin da subito che non sarà possibile ripararli.
Controllo successivamente che il nodo non sia già sano, cioè che non c'è bisogno di ripararlo.
 di riparare il nodo $n$, cerco tra i suoi nodi adiacenti un nodo $r$ con le seguenti proprietà:
\begin{itemize}
    \item il nodo $n$ deve poter ricevere (o inviare) del flusso dal nodo $r$ e dall'arco che li collega;
    \item il nodo $r$ deve essere valido, quindi SinkValid o SourceValid a seconda della parte che esploro;
    \item il nodo $r$ deve avere label precedente a quella di $n$, cioè $r.label +1 = n.label$, questo non vale se sto trattando un nodo di confine e devo esplorare i nodi dalla parte di source.
\end{itemize}
Se il nodo $r$ ha tutte queste tre proprietà,allora posso procedere a aggiornare i dati e indicare che il nodo è guarito, altrimenti indico che il nodo non è valido per la parte esplorata.

\subsection{SickPropagation}
Si divide in quella per i nodi di source e in quella per i nodi di sink.
Le differenze sono minime : in quale coda vado a inserire i nodi, come si deve comportare la repair e quali nodi devo analizzare, dalla parte di source i nodi esplorati da $s$ e i nodi di confine, dall'altra parte i nodi esplorati a partire da $t$.
Ricevo in input il nodo non valido che deve propagare la malattia e la coda (o meglio, il puntatore della coda).
creo una coda di nodi, dove andrò a inserire i nodi malati.
Inserisco per primo il nodo ricevuto in input.
Finché la coda non è vuota,procedo come segue:
\begin{enumerate}
    \item estraggo il nodo malato dalla coda;
    \item provo a ripararlo;
    \item nel caso riesca a ripararlo, controllo se è un nodo di confine o meno, nel caso non lo sia, lo inserisco nella coda ricevuta in input, altrimenti lo restituisco;
    \item se non riesco a ripararlo, inserisco i nodi che lo avevano come predecessore (esplorati da $s$), o come successore (esplorati da $t$), nella coda dei nodi malati.
\end{enumerate}  


\end{document}
