\documentclass{article}
\usepackage[utf8]{inputenc}
\usepackage{algorithm}
%\usepackage{algorithmic}
\usepackage{algcompatible}
\usepackage{amsmath}

\title{algoritmi bidirezionali}
\author{Filippo Magi }

\begin{document}

\maketitle

\section{Ottimizzazione sono nelle ultime label}
\begin{algorithm}
    \caption{Ricerca del massimo flusso ricalcolando solo nelle ultime label}
    \begin{algorithmic}
        \REQUIRE Rete $(G,u,s,t)$ .
        \ENSURE valore del flusso massimo
        \STATE $noCap \leftarrow$ null
        \STATE $malati \leftarrow$ pila di nodi vuota
        \WHILE{TRUE}
        \STATE $f \leftarrow$ DoBfs($G,noCap$)
        \IF{ $f = 0$}
        \STATE \textbf{break}
        \ENDIF
        \STATE $mom \leftarrow t$
        \WHILE{ $mom \ne s$}
        \STATE $mom.PreviousEdge.AddFlow(t.flussoPassante)$
        \IF{$u(mom.PreviousEdge) = 0$}
        \STATE $malati.$push$(mom)$
        \ENDIF
        \STATE $mom \leftarrow mom$.previousNode
        \ENDWHILE
        \ENDWHILE
        \STATE \textbf{return} $s$.FlussoUscente
    \end{algorithmic}
\end{algorithm}


\begin{algorithm}
    \caption{Algoritmo DoBfs con ottimizzazione solo nelle ultime label}
    \begin{algorithmic}
        \REQUIRE rete $(G,u,s,t)$, pila di nodi $malati$
        \ENSURE valore del flusso inviato a $t$, $N(G)$ aggiornati con informazioni sul percorso da fare
        \STATE $coda \leftarrow$ coda vuota di nodi
        \IF{$malati.isEmpty$ }
        \STATE $coda.$Enqueue$(s)$
        \FORALL{$n \in V(G)$}
        \STATE $n.$Reset()
        \ENDFOR
        \ELSE
        \STATE $repaired \leftarrow$ true
        \STATE $ p \leftarrow $ null
        \WHILE{$\neg malati.isEmpty$}
        \STATE $n \leftarrow malati.$pop()
        \STATE UpdateInFlow($n,p$) \COMMENT{se $p \ne null$, aggiorno i nodi tra p ed n con il nuovo valore di InFlow, passando da 0 a un valore positivo}
        \STATE Repair($n$)\Comment{controllo se c'è un nodo con label = noCap.label-1 e con capacità o flusso diversa da 0}
        \IF{non sono riuscito a riparare $n$}
        \STATE $repaired \leftarrow $false
        \STATE \textbf{break}
        \ENDIF
        \STATE $p \leftarrow n$
        \ENDWHILE
        \IF{$repaired$}
        \STATE $f \leftarrow $ UpdateInFlow $(t,n)$
        \STATE \textbf{return} f
        \ENDIF
        %capire se quello che ho scritto è giusto o meno
        \STATE coda $\leftarrow x \in V(G) \|  x.label = (n.label - 1)$
        \FORALL{$x \in V(G) \| x.label \ge n.label$}
        \STATE $x$.Reset()
        \ENDFOR
        \ENDIF
        \algstore{finishedInit}
    \end{algorithmic}
\end{algorithm}
\newpage
\begin {algorithm}
\begin{algorithmic}
    \algrestore{finishedInit}
    \WHILE{la coda non è vuota}
    \STATE $element  \leftarrow coda$.dequeue()
    \IF{$element$.valid} \Comment{label valida}
    \FORALL{$edge \in element.Edges$}
    \STATE $n \leftarrow edge.$nextNode
    \STATE $p \leftarrow edge$.previousNode
    \IF{ $p = element \land u_t(edge) > 0 \land (\neg n.visited \lor \neg n.valid$)} \COMMENT{per capire se il nodo è stato visitato o meno, controllo che $inflow > 0$}
    \STATE $n$.update(p,edge) \COMMENT{label, flusso entrante e nodo precedente, nel caso sia necessario "riparo" il nodo}
    \STATE $edge$.Reversed = false \Comment{per capire se devo inviare o ricevere flusso}
    \IF {$n = t$}
    \STATE \textbf{return} $n$.flussoPassante
    \ELSE
    \STATE $coda$.Enqueue($n$)
    \ENDIF
    \ELSIF{$n = element \land f(edge) > 0 \land (\neg p.visited \lor \neg p.valid)$}
    \STATE $p$.update(n,edge)
    \STATE $edge$.reversed = true
    \IF{$p = t$}
    \STATE \textbf{return} $p$.flussoPassante
    \ELSE
    \STATE $coda$.Enqueue($p$)
    \ENDIF
    \ENDIF
    \ENDFOR
    \ENDIF
    \ENDWHILE
    \STATE \textbf{return} 0
\end{algorithmic}
\end{algorithm}
\end{document}