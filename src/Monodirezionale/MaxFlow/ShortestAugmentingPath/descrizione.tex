\documentclass{article}
\usepackage[utf8]{inputenc}
\usepackage{amsmath}

\title{presentazione algoritmo ShortestAugemntingPath}
\author{Filippo Magi }

\begin{document}

Algoritmo ricorsivo che cerca il flusso massimo tramite una ricerca in profondità.

\section{Strutture dati utilizzate}

\subsection{BiEdge}

I nodi vengono collegati tra di loro da archi BiEdge, che contiene le informazioni da quale nodo esce e in quale nodo entra.
Ovviamente conserva in memoria la quantità di flusso che passa e la sua capacità residua.
Durante l'inizializzazione, qualsiasi nodo tranne il nodo destinazione $t$, avrà distanza pari a -1, per indicare che non è stato ancora esplorato.

\subsection{Node}

Ovviamente ogni nodo tiene una lista con tutti gli archi a lui collegati, si salva il nodo e l'arco predecessore per indicazioni su dove inviare il flusso.
Inoltre, si salva la distanza tra quel nodo e $t$.

\subsection{Graph}

Un insieme di nodi.

\section{ Descrizione algoritmo}

Eseguo una bfs, esplorando tutti i nodi.
Invio il flusso che ho trovato grazie a lei.
cancello le informazioni riguardanti le indicazioni su tutti i nodi (elimino le informazioni su previousNode e previousEdge).
dopo aver ottenuto le distanze, finché la distanza tra s è minore del numero di nodi del grafo, proseguo con l'algoritmo ricorsivo di ShortestAugmentingPath (nel codice e pseudo-codice, Dfs), e con il percorso trovato, invio il flusso massimo consentito.

\subsection{bfs}

iniziamo con una bfs da $t$ esplora tutti i nodi, indicando la distanza da lui, oltre a trovare un percorso da $t$ e il nodo sorgente $s$.

\subsection{dfs}

Algoritmo ricorsivo che cerca di procedere da s verso t, attraverso procede, cioè se il nodo \textbf{p} che sto analizzando ha un arco che lo collega con un nodo \textbf{n} tale che \textbf{p}.distance = \textbf{n}.distance +1, e con capacità positiva, allora procedo ad salvare in una apposita coda e analizzare \textbf{n},  a meno che non sia  il nodo cercato $t$ (che verrà comunque salvato nella coda).
Se non è possibile procedere, procedo a fare il retreat, cioè cerco tra i nodi che escono da \textbf{p} quello con distanza minore, e rendo la distanza di \textbf{p} il valore di quella distanza +1 ,per poi analizzare il nodo che avevo analizzato antecedente, nel caso il non lo abbia (cioè \textbf{p} è il nodo sorgente $s$), ripeto con \textbf{p}.
Appena trovo il nodo t, invio il flusso trovato.
Per ogni nodo esplorato, lo inserisco in una coda, in maniera tale, dopo aver inviato il flusso, di cancellare tutti i dati dei nodi esplorati.

\end{document}
