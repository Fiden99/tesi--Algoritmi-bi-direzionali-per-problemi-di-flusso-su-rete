\documentclass{article}
\usepackage[utf8]{inputenc}
\usepackage{algorithm}
\usepackage{algorithmic}
\usepackage{amsmath}

\title{Shortest Augmenting Path}
\author{Filippo Magi }

\begin{document}

\maketitle

\section{Shortest Augmenting Path}
\begin{algorithm}
\caption{Shortest Augmenting Path}
\begin{algorithmic}
\REQUIRE grafo dei reisidui $\overset{\leftrightarrow}{G}$
\ENSURE flusso massimo inviato, $\overset{\leftrightarrow}{G}$ aggiornato
\STATE s $\leftarrow$ nodo sorgente di $\overset{\leftrightarrow}{G}$
\STATE t $\leftarrow$ nodo destinazione di $\overset{\leftrightarrow}{G}$
\STATE eseguo bfs, che parte da t, assegnando a ogni nodo la distanza da t 
\STATE invio flusso nel percorso deciso dalla bfs
\STATE flussoInviato $\leftarrow$ flusso inviato (dal percorso deciso dalla bfs)
\WHILE{distanza tra t e s < numero di nodi di $\overset{\leftrightarrow}{G}$}
\STATE f $\leftarrow$ Dfs$(\overset{\leftrightarrow}{G}, s, +\infty)$
\IF{f $\ne 0$}
\STATE invia flusso f nel grafo nel percorso indicato
\ELSE
\STATE \textbf{break}
\ENDIF
\STATE flussoInviato $\leftarrow$ f + flussoInviato 
\ENDWHILE
\RETURN flussoInviato
\end{algorithmic}
\end{algorithm}
\begin{algorithm}
\caption{Dfs per trovare il flusso massimo in Shortest Augmenting Path}
\begin{algorithmic}
\REQUIRE grafo dei residui  $\overset{\leftrightarrow}{G}$,Nodo start, valore f
\ENSURE valore del flusso inviabile, percorso percoribile per poter inviare il flusso prima indicato
\IF{distanza tra nodo start e t $<$ numero dei nodi presenti in $\overset{\leftrightarrow}{G}$}
\FORALL{arco edge che entra o esce nel nodo start}
\IF{edge è un arco uscente da start AND edge è ammissibile (distanza tra i nodi = 1 e capacità residua positiva)}
\STATE arco precedente di n $\leftarrow$ edge ( di conseguenza salvo anche il nodo precedente)
\STATE f $\leftarrow \min$(f,capacità di edge)
\IF{n è il nodo di destinazione di $\overset{\leftrightarrow}{G}$}
\RETURN f
\ENDIF
\RETURN Dfs($\overset{\leftrightarrow}{G}$,n,f)
\ENDIF
\ENDFOR

\STATE min $\leftarrow +\infty$
\FORALL{arco edge che esce da start}
\IF{capacità di edge è positiva}
\STATE min $\leftarrow \min$(min, distanza del nodo entrante di edge)
\ENDIF
\ENDFOR
\STATE distanza tra start e t $\leftarrow$ min
\IF{start è nodo sorgente di $\overset{\leftrightarrow}{G}$}
\RETURN Dfs($\overset{\leftrightarrow}{G}$,start,f)
\ELSE
\STATE m $\leftarrow$ precedessore di start
\STATE cancello dati contenuti in m (tranne distanza)
\RETURN Dfs($\overset{\leftrightarrow}{G}$,m,f)
\ENDIF
\ENDIF
\RETURN 0
\end{algorithmic}
\end{algorithm}
\end{document}