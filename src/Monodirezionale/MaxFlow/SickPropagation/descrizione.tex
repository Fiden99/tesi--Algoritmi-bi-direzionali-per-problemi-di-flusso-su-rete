\documentclass{article}
\usepackage[utf8]{inputenc}
\usepackage{algorithm}
\usepackage{algcompatible}
\usepackage{amsmath}

\title{Presentazioni dell'algoritmo di ricerca del flusso massimo monodirezionale con propagazione della malattia  }
\author{Filippo Magi }

\begin{document}

\maketitle

\section{Strutture dati}
\subsection{BiEdge}
I nodi vengono collegati tra di loro da archi BiEdge, che contiene le informazioni da quale nodo esce e in quale nodo entra.
Ovviamente conserva in memoria la quantità di flusso che passa e la sua capacità residua, oltre a un booleano (reversed) che mi indica che quel arco, durante l'invio del flusso, dovrà inviarlo o ritirarlo.

\subsection{Node}

Node contiene al suo interno informazioni sugli archi a lui collegati (Lista di BiEdge),
un booleano che indica se il nodo è valido o meno,
le indicazioni sul percorso che deve fare, quindi sia previousNode e previousEdge,
la label, cioè la distanza tra lui e il nodo sorgente \textit{s},
oltre a un booleano che mi indica se è stato visitato o meno.
\subsection{Graph}
Contiene un insieme per i nodi non validi
e una lista di insiemi, tale che ogni insieme contiene i nodi
con una certa label.

\section{Descrizione}

il funzionamento è molto simile a quello di LastLevelOpt,
con una differenza in fase di "inizializzazione".
Provo a riparare tutti i nodi malati di \textbf{noCaps},
se riesco a ripararli tutti, procedo dichiarando di aver trovato il percorso,
nel caso in cui un nodo non sia riparabile, lo inserisco in una coda di \textbf{malati}.
Per ogni nodo $\in$ \textbf{malati}, procedo a sottoporlo a SickPropagation,
nel caso riesco a trovare un percorso, restituisco il valore del flusso inviabile, altrimenti procedo con il nuovo nodo.
Se nessuna propagazione della malattia mi ha portato a un percorso, controllo se \textit{t} è valido, nel caso lo sia, ho automaticamente trovato un percorso.
Nel caso in cui \textit{t} non sia più valido cancello tutte le informazioni dai nodi con label pari o maggiore
a quella del primo nodo \textbf{first} esplorato dalla coda di \textbf{malati} .
Nel caso in cui SickPropagation non abbia già inserito dei nodi in coda, vi inserisco i nodi $n | n.label = first.label -1$,
poi procedo con la ricerca di \textit{t} normalmente, come in LastLevelOpt.

\subsection{SickPropagation}

Datomi il nodo $n$, lo inserisco in una coda $malati$.
Provo a ripararlo, nel caso in cui riesca, lo inserisco nella coda, che verrà poi utilizzata per la ricerca del cammino, a meno che non sia il nodo destinazione \textit{t}, in tal caso restituisco il valore del flusso inviabile attraverso il cammino descritto dai previousNode.
Nel caso non sia possibile ripararlo,inserisco tutti i nodi adiacenti che sono stati precedentemente esplorati da lui, cioè con $malati.enqueue( x \in A(n) | x.previousNode = n)$.
Ripeto finché o riparo il nodo $t$, o si esaurisce la coda di $malati$.


\end{document}